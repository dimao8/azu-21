\documentclass[12pt,a4paper]{article}
\usepackage[utf8]{inputenc}
\usepackage[russian]{babel}
\usepackage[T2A]{fontenc}
\usepackage{graphicx}
\usepackage{indentfirst}
\usepackage{hyperref}
\usepackage{amsmath}

\author{Хрущев Дмитрий aka DimaO}
\title{ШИМ в проекте АЗУ-21}
\date{2022}

\begin{document}
\maketitle
\newpage

Частота микроконтроллера STM32F030 выбрана 8~МГц, поскольку используется внутренний генератор. Тактирование таймеров выбрано напрямую от генератора, поскольку не используется модуль ФАПЧ. Следовательно частота тактирования модуля таймера TIM3 также 8~МГц. Введем обозначения: $F_{OSC}$ -- частота осциллятора 8~МГц, $F_{TIM3}$ -- частота тактирования таймера, также 8~МГц.

Условия задачи поставлены так: необходимо разработать конфигурацию таймера так, чтобы на выходе мы получили частоту не выше 1000~Гц. Ширина импульса не должна превышать $1/6$ периода ШИМ. Необходимо обеспечить как минимум 100 градаций изменения ширины импульса.

Введем дополнительные обозначения: $D_{PSG}$ -- делитель таймера, $F_{PWM}$ -- частота ШИМ, $T_{PWM}$ -- длительность одного кванта ШИМ.

Итак, возьмем для крайнего случая частоту ШИМ $F_{PWM}=10~\mbox{кГц}$. Следовательно, полный коэффициент деления должен быть
\begin{equation}
D=\frac{F_{TIM3}}{F_{PWM}}=\frac{8~\mbox{МГц}}{10~\mbox{кГц}}=800.
\end{equation}

Теперь из этого числа необходимо извлечь кванотование, умноженное на коэффициент $\left(1/6\right)^{-1}=6$. Если предположить минимальное квантование в 100 единиц, то коэффициент получается 600. Делим делитель на 600. $\frac{800}{600}=1,\left[3\right]$. Разумеется нельзя установить дробный коэффициент деления. Поскольку мы использовали частоту ШИМ в 10~кГц, что находится на максимальном крае, мы можем пожертвовать скоростью в пользу быстродействия обслуживания таймера. В этом случае берем коэффициент $D_{PSG}=2$. Пересчитаем частоту ШИМ.

\begin{equation}
F_{PWM}=\frac{F_{TIM3}}{D_{PSG}\cdot 600}=6666,[6].
\end{equation}

Такой результат вполне удовлетворителен, поскольку нижняя граница частоты ШИМ -- 1~кГц. Результат выше в 6,6 раза.

Необходимо, однако, понимать, что при выставлении минимальной ширины импульса в 1 квант, потенциально прерывания от таймера будут поступать с интервалами 150~мкс и 250~нс. И, если первый достаточно большой для обработки, то второй, скорее всего, никогда обработам не будет, поскольку на обработку прерывания предоставляется всего 2 такта. Решить проблему можно увеличением частоты тактирования процессора.

\end{document}